Slide 7

1.1 Design outputs (econometric outputs and academic/policy products)

1.2 Design variable descriptions (defines the variables to be collected  as well as any other variables that will be derived from them in sub-process 5.5)

1.3 Design collection (determines the most appropriate data (census, sample survey, or other) and mechanism to access data) [`r icons::icon_style(icons::fontawesome("github"))` Issue #3](https://github.com/hbaraho/cuposBEA/issues/3)


Slide 8
This phase describes the processing of input data and their preparation for analysis. It is made up of sub-processes that integrate, classify, check, clean, and transform input data, so that they can be analysed and disseminated as statistical outputs. For statistical outputs produced regularly, this phase occurs in each iteration. The sub-processes in this phase can apply to data from both statistical and non-statistical sources.

# Subprocess
![Image](https://user-images.githubusercontent.com/56052549/188322464-3adfbeca-a070-4321-9219-f4576bd12414.png)

- [ ] Integrate data
- [ ] Classify and code
- [ ] Review and validate
- [ ] Edit and impute
- [ ] Construct new variables and units
- [ ] Calculate weights (survey data )
- [ ] Calculate aggregates
- [ ] Finalise data files

# Data input

## **1.  Centralized admission process information**

- [ ] Studdets performance on the national university entrance exam 
          - [ ]  Socioeconomic background (self-reported family income, parent's education, municipality in which students lives)
          - [ ]  Seat type requested
- [ ] Admission offers from institutions (universities characteristics, number of vacancies, weights, tuition, duration, program allocation)
- [ ] Student preference list
- [ ] Final assignament and enrollment decisions 

## **2. Survey data**
- [ ] Student perceptions and preferences for programs

## **3.Academic performance in college**
- [ ] GPA
- [ ] Desertion and academic progress (years)

## **4. Labor market**
- [ ] Labor market prospects of each program (average wages, employment probability one year after graduation, evolution)

Note: The "Process"  and "Analyse" phases can be iterative and parallel. Analysis can reveal a broader understanding of the data, which might make it apparent that additional processing is needed. 
